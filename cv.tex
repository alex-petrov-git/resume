% use XeLaTeX

\documentclass[a4paper,12pt]{article}
\usepackage{graphicx}
\usepackage{geometry}
\usepackage{multicol}
\usepackage[russian]{babel}
\usepackage{parskip}
\usepackage{titlesec}
\usepackage{xcolor}
\usepackage{fontspec}
\usepackage{caption}
\usepackage{fontawesome5}
\usepackage{amssymb}
\usepackage{setspace}
\usepackage{booktabs}
\usepackage{tablefootnote}
\usepackage[most]{tcolorbox}
\usepackage{tikz}
\usetikzlibrary{shapes}





\setmainfont{Calibri}
\geometry{top=0.7cm, bottom=1cm, left=1.5cm, right=1.5cm}
% Тёмная тема: фон и цвета
\pagecolor[RGB]{30,30,30} % Тёмно-серый фон
\color{white} % Основной текст белый
% Настройка стилей заголовков
\titleformat{\section}{\Large\bfseries\color{violet}}{\thesection}{1em}{}[\titlerule]
\titleformat{\subsection}{\large\bfseries\color{violet}}{\thesection}{1em}{}[\titlerule]
\titleformat{\subsubsection}{\bfseries\color{violet}}{\thesection}{1em}{}[]
% Цвета для тёмной темы
\definecolor{textblue}{RGB}{173, 216, 230} % Светло-голубой для имени
\definecolor{lightblue}{RGB}{50,50,50} % Тёмный фон для контактов
\definecolor{linkcolor}{RGB}{200, 162, 200} % Светло-фиолетовый для ссылок
\definecolor{linkvisited}{RGB}{221, 160, 221} % Ещё светлее для посещенных
\definecolor{violet}{RGB}{200, 162, 200} % Фиолетовый для акцентов
\definecolor{purple}{RGB}{200, 162, 200} % То же для совместимости
\definecolor{pagecolor}{RGB}{30, 30, 30}

\setlength{\parindent}{0pt}
\setlength{\parskip}{6pt}
\setlength{\columnsep}{0cm}
\pagestyle{empty}
\usepackage{hyperref}
\hypersetup{colorlinks=true,linkcolor=linkcolor,filecolor=linkcolor,urlcolor=linkcolor,citecolor=linkcolor,pdftitle={Резюме Петрова Александра Глебовича},pdfpagemode=FullScreen}
% Стиль блока для аннотации в тёмной теме (компактно)
\tcbset{myblock/.style={colframe=violet,colback=pagecolor,coltext=white,left=10pt,sharp corners,before skip=10pt,after skip=10pt,boxrule=0.3mm}}
\setcounter{secnumdepth}{0}
\begin{document}
	%\colorbox{lightblue}{%
		\raggedright % For fixing overfull/underfull hbox
		%\hspace{0.3cm}
		\begin{minipage}{0.65\textwidth} % латинская 'c'
			\vspace{0.5cm}
			
			{\fontsize{40}{40}\selectfont\color{textblue}\textbf{Петров Александр}}
			
			\vspace{0.3cm} 
			
			{\fontsize{40}{40}\selectfont\color{textblue}\textbf{Глебович}}
			\vspace{0.6cm}
			
			% блок контактов
			\begin{tabular}{@{}l l l l@{}}
				\faMapMarker\ & \raisebox{0.1ex}{{\href{https://yandex.ru/maps/-/CHtXE-JQ}{Москва, Зюзино}}} 
				& \faEnvelope\ & \raisebox{0.2ex}{\href{mailto:petrov.aleksandr@phystech.edu}{petrov.aleksandr@phystech.edu}} \\
			\end{tabular}
			
			\vspace{0.2cm}
			\begin{tabular}{@{}l l l l l l@{}}
				\faPhone\ & \href{tel:+79251457290}{+79251457290} 
				& \faTelegram\ & \raisebox{0.2ex}{\href{https://t.me/petrovortex}{@petrovortex}} & \faGithub\ & \raisebox{0.2ex}{\href{https://github.com/alex-petrov-git}{alex-petrov-git}} \\
			\end{tabular}
		\end{minipage}
		\hfill
		\begin{minipage}{0.315\textwidth}
			\vspace{0pt}
			\raggedleft
			\begin{tikzpicture}
				\node[regular polygon,
				regular polygon sides=6,
				draw=violet,
				line width=0.8mm,
				fill=lightblue,
				inner sep=0pt,
				minimum size=5.5cm,
				path picture={
					\node at (path picture bounding box.center){
						\includegraphics[width=3.3cm]{photo.png}
					};
				}] {};
			\end{tikzpicture}
		\end{minipage}
		
	%}
	\vspace{0.5cm}
	
	% Рамка с аннотацией
	\begin{tcolorbox}[myblock]
		\setstretch{1.5}\selectfont\fontsize{15}{15}\textit{В этом резюме я собрал общую информацию о себе, своих достижениях и проектах, которыми я горжусь!}
	\end{tcolorbox}
	\vspace{5pt}
	% Само резюме без blank lines
	
	\subsection*{What I do}
	Сейчас\footnote{Сентябрь 2025} я изучаю NLP и Speech Processing, а также углубляю знания по базовому ML и теории вероятностей. Прохожу курсы
	\begin{itemize}
	\item[\faArrowRight] DLS-2,3: \href{https://stepik.org/course/251729/syllabus}{\textcolor{violet}{\textbf{NLP}}}\footnote{Здесь и далее присутствуют ссылки, на них можно тыкнуть.}, \href{https://stepik.org/course/251868/syllabus}{\textcolor{violet}{\textbf{Speech}}}
	\item[\faArrowRight] \href{https://ods.ai/tracks/nlp-course-autumn-2025}{\textcolor{violet}{\textbf{ODS NLP}}}
	\item[\faArrowRight] \href{http://www.machinelearning.ru/wiki/index.php?title=%D0%9C%D0%B0%D1%88%D0%B8%D0%BD%D0%BD%D0%BE%D0%B5_%D0%BE%D0%B1%D1%83%D1%87%D0%B5%D0%BD%D0%B8%D0%B5_%28%D0%BA%D1%83%D1%80%D1%81_%D0%BB%D0%B5%D0%BA%D1%86%D0%B8%D0%B9%2C_%D0%9A.%D0%92.%D0%92%D0%BE%D1%80%D0%BE%D0%BD%D1%86%D0%BE%D0%B2%29}{\textcolor{violet}{\textbf{Математические Основы Машинного Обучения}}} (МФТИ, курс Воронцова К.В.)
	\item[\faArrowRight] \href{https://mccme.ru/ru/nmu/courses-of-nmu/osen-20252026/nmu_autumn2025_terver/}{\textcolor{violet}{\textbf{Теория вероятностей}}} (НМУ, курс Шапошникова С.В.)	
	\end{itemize}

	
	\subsection*{Образование}
	\textcolor{violet}{\textbf{Высшее}} (бакалавриат): 
	\begin{itemize}
		\item[\faCheck] \textcolor{violet}{\textbf{МФТИ, 2023}},
		\item[\faCheck] направление \textit{прикладные математика и физика}, 
		\item[\faCheck] средний балл 8/10, "отлично"\ (подробнее см. Дополнение).
	\end{itemize}
	
	\textcolor{violet}{\textbf{Высшее}} (магистратура):
	\begin{itemize}
		\item[\faCheck] \textcolor{violet}{\textbf{МФТИ, 2025}},
		\item[\faCheck] направление \textit{прикладные математика и физика}, 
		\item[\faCheck] средний балл 8.6/10, "отлично"\ (подробнее см. Дополнение).
	\end{itemize}
	
	\textcolor{violet}{\textbf{Дополнительное}}: Цифровая кафедра МФТИ, курс \textit{Основы глубокого обучения}, \textcolor{violet}{\textbf{2025}}.
	
	\textcolor{violet}{\textbf{Дополнительное}}: Deep Learning School (DLS), \href{https://stepik.org/course/230362/syllabus}{\textcolor{violet}{\textbf{первый семестр}}}, \textcolor{violet}{\textbf{весна 2025}}.
	
	В рамках дополнительного образования познакомился с: 
	\begin{itemize}
		\item[\faCheck] Классической теорией машинного обучения: оптимизационная и вероятностная постановки задачи машинного обучения, регрессия, классификация, дерево решений, метод опорных векторов, ансамблирование,
		\item[\faCheck] Основами глубокого обучения: MLP, CNN, RNN, обучение методом обратного распространения градиента, эвристики для метода стохастического градиента, (RMSprop, Adam, AdamW,...), автоэнкодеры, генеративно-состязательные сети (GANs),
		\item[\faCheck] Решением задач компьютерного зрения: классификация, сегментирование и детектирование.
	\end{itemize}
	
	\subsection*{Опыт работы}
	
	\textcolor{violet}{\textbf{Инженер}} в Центральном аэрогидродинамическом институте (ЦАГИ), \textcolor{violet}{\textbf{2021-2025}}. Исследовал численно и теоретически механизмы излучения аэродинамического звука в горячих потоках.
	
	
	\textcolor{violet}{\textbf{Преподаватель олимпиадной физики}} в МФТИ, \textcolor{violet}{\textbf{2023}}. 
	
	\subsection*{Достижения}
	\begin{itemize}
	\item[\textcolor{violet}{\faTrophy}] Абсолютная победа в \href{https://www.figma.com/board/JqewkyeKxGNv3K8oA2SAdw}{\textcolor{violet}{\textbf{хакатоне}}} ЦК МФТИ, кейс \textit{Цифриум}, команда \textit{Конспектум}, \textcolor{violet}{\textbf{2025}}.
	
	Необходимо было создать сервис генерирующий конспект для учебного видео. В результате был создан \href{https://github.com/alex-petrov-git/konspektum}{\textcolor{violet}{\textbf{телеграм-бот}}}. 
	
	\item[\textcolor{violet}{\faTrophy}] Победитель олимпиады \textcolor{violet}{\textbf{Я-Профессионал}} по математическому моделированию, \textcolor{violet}{\textbf{2025}}.
	
	\item[\textcolor{violet}{\faRocket}] Создание и развитие телеграм-бота \href{https://t.me/faltuna_bot}{\textcolor{violet}{\textbf{Фалтуна}}}, \textcolor{violet}{\textbf{2025-н.в.}}
	
	Я создатель общего телеграм-чата (на 360 человек) моего факультета. Осознав сходство этого чата и твиттера я решил, что ему нужен свой Грок. Так была создана Фалтуна. На данный момент её основные функции: регистрация участников чата, модерация, канал связи с администрацией (теперь не студенты пересылают сообщения, а Фалтуна). 
	
	В ближайшем будущем планируется добавить: функционал записи в стиралку и к плотнику/сантехнику, сервис Random Coffee, помощь по учебным и бюрократическим вопросам.
	
	\item[\textcolor{violet}{\faRocket}] Развитие проекта \href{https://vk.com/botay_falt}{\textcolor{violet}{\textbf{BOTAY FALT}}}, \textcolor{violet}{\textbf{2020-2021}}.
	
	Этот проект был предназначен для агрегации учебных материалов в одном месте. Когда-то у Физтеха был бесконечный гугл-диск и на нем было удобно все складировать. Вокруг такого гугл-диска была создана группа вк, где публиковались различные подборки учебных материалов. Сейчас этот проект в заморозке, но в планах есть его возрождение в виде сервиса Фалтуны (stay tuned).
	
	\item[\textcolor{violet}{\faHeart}] Руководство \href{https://t.me/falt_basket}{\textcolor{violet}{\textbf{секцией баскетбола}}}, \textcolor{violet}{\textbf{2019-2024}}.
	
	Секция баскетбола это мое место силы на Физтехе. Здесь я не только отдыхал от учебы, но и познакомился с кучей крутых ребят, а также впервые почувствовал себя в роли лидера команды (и вырос как личность, я считаю).
	
	До 2023 года мы занимались без тренера и просто играли. Однако, после досадного поражения на соревнованиях нам захотелось играть лучше, а для этого 100\% нужен тренер. Мне, как старосте, необходимо было решить этот вопрос. 
	
	TL;DR\\ Я нашел спонсоров и тренера, мы отремонтировали площадку, закупили новый инвентарь, осенью провели крутой открытый турнир на котором собрали новую команду факультета, а весной мы заняли 2-е место на межфакультетских соревнованиях МФТИ.
	
	\end{itemize}
	
	
	
	
	\newpage 
	
	\subsection*{Дополнение}
	
	\subsubsection*{Высшая математика}
	{\fontsize{9}{11}\selectfont
	\begin{tabular}{p{0.7\textwidth} p{0.05\textwidth} p{0.08\textwidth}}
	\toprule
	\textbf{Дисциплина} & \textbf{З.Е.} & \textbf{Оценка} \\
	\midrule
	Аналитическая геометрия & 3 & Отлично \\
	Введение в математический анализ & 6 & Отлично \\
	Гармонический анализ & 3 & Отлично \\
	Дискретная математика & 3 & Отлично \\
	Дифференциальные уравнения & 6 & Отлично \\
	Кратные интегралы и теория поля & 3 & Отлично \\
	Линейная алгебра & 3 & Отлично \\
	Математические методы планирования и интерпретации эксперимента & 4 & Отлично \\
	Многомерный анализ, интегралы и ряды & 6 & Отлично \\
	Теория вероятностей & 2 & Отлично \\
	Теория функций комплексного переменного & 4 & Отлично \\
	Уравнения математической физики & 7 & Отлично \\
	\bottomrule
	\end{tabular}}
	
	\subsubsection*{Общая и теоретическая физика}
	{\fontsize{9}{11}\selectfont
	\begin{tabular}{p{0.7\textwidth} p{0.05\textwidth} p{0.08\textwidth}}
	\toprule
	\textbf{Дисциплина} & \textbf{З.Е.} & \textbf{Оценка} \\
	\midrule
	Аналитическая механика & 7 & Отлично \\
	Квантовая механика & 4 & Хорошо \\
	Кинетическая теория газов & 3 & Отлично \\
	Колебания и волны & 2 & Хорошо \\
	Общая физика: квантовая физика & 2 & Хорошо \\
	Общая физика: лабораторный практикум & 15 & Хорошо \\
	Общая физика: механика & 4 & Хорошо \\
	Общая физика: оптика & 4 & Отлично \\
	Общая физика: термодинамика и молекулярная физика & 4 & Отлично \\
	Общая физика: электричество и магнетизм & 5 & Отлично \\
	Основы современной физики & 3 & Удовл. \\
	Основы современной физики: лабораторный практикум & 2 & Хорошо \\
	Статистическая физика & 3 & Хорошо \\
	Теория поля & 2 & Отлично \\
	\bottomrule
	\end{tabular}}
	
	\subsubsection*{Компьютерные науки}
	{\fontsize{9}{11}\selectfont
	\begin{tabular}{p{0.7\textwidth} p{0.05\textwidth} p{0.08\textwidth}}
	\toprule
	\textbf{Дисциплина} & \textbf{З.Е.} & \textbf{Оценка} \\
	\midrule
	Вычислительная математика & 6 & Хорошо \\
	Вычислительные методы в механике & 2 & Отлично \\
	Информатика & 7 & Отлично \\
	Компьютерные технологии & 6 & Отлично \\
	Метод конечных элементов в задачах вычислительной аэродинамики & 2 & Отлично \\
	Нейросетевые технологии и robust-оптимизация в задачах аэродинамики & 3 & Отлично \\
	Цифровая обработка сигнала и теория управления & 5 & Хорошо \\
	\bottomrule
	\end{tabular}}

	\subsubsection*{Общеобразовательные дисциплины}
	{\fontsize{9}{11}\selectfont
		\begin{tabular}{p{0.7\textwidth} p{0.05\textwidth} p{0.08\textwidth}}
			\toprule
			\textbf{Дисциплина} & \textbf{З.Е.} & \textbf{Оценка} \\
			\midrule
			Безопасность жизнедеятельности & 1 & Зачтено \\
			Быть зрителем & 3 & Отлично \\
			История & 1 & Хорошо \\
			Логика и аргументация & 2 & Отлично \\
			Основы политологии & 1 & Хорошо \\
			Правоведение & 1 & Отлично \\
			Философия & 1 & Хорошо\footnotemark  \\
			Экономика & 2 & Отлично \\
			\bottomrule
	\end{tabular}}

	\footnotetext[1]{Я получил хор по философии потому, что наш преподаватель, который считает что человечество станет счастливее если уйдет в пещеры, также считает, что "Гарри Поттер и Методы Рационального Мышления"\ Юдковского нельзя было использовать при написании реферата, потому что это не философский текст, а фигня. По политологии и истории похожая ситуация.}
	
	\subsubsection*{Английский язык и физическая культура}
	{\fontsize{9}{11}\selectfont
	\begin{tabular}{p{0.7\textwidth} p{0.05\textwidth} p{0.08\textwidth}}
	\toprule
	\textbf{Дисциплина} & \textbf{З.Е.} & \textbf{Оценка} \\
	\midrule
	Иностранный язык (Английский язык, уровень В2) & 17 & Хорошо \\
	Иностранный язык (Английский язык, Межкультурная коммуникация) & 4 & Хорошо \\
	Физическая культура & 2 & Зачтено \\
	\bottomrule
	\end{tabular}}
	
	\subsubsection*{Специальные курсы (аэродинамика и смежные дисциплины)}
	{\fontsize{9}{11}\selectfont
	\begin{tabular}{p{0.7\textwidth} p{0.05\textwidth} p{0.08\textwidth}}
	\toprule
	\textbf{Дисциплина} & \textbf{З.Е.} & \textbf{Оценка} \\
	\midrule
	Акустика кабин летательных аппаратов & 2 & Отлично \\
	Асимптотическая теория отрыва & 2 & Отлично \\
	Аэроакустика & 4 & Отлично \\
	Аэродинамика больших скоростей & 3 & Отлично \\
	Аэродинамика органов управления & 1 & Хорошо \\
	Аэродинамическое нагревание & 3 & Хорошо \\
	Введение в аэродинамику & 3 & Отлично \\
	Введение в физику полета и авиационные технологии & 1 & Отлично \\
	Вихревые и отрывные течения & 5 & Отлично \\
	Гидродинамика морских летательных аппаратов & 2 & Отлично \\
	Гидродинамическая устойчивость & 4 & Отлично \\
	Динамика вязкого газа & 6 & Отлично \\
	Динамика полета & 2 & Хорошо \\
	Динамика разреженного газа & 3 & Отлично \\
	Основы инженерной подготовки & 5 & Отлично \\
	Основы прочности & 3 & Отлично \\
	Основы турбулентного пограничного слоя & 2 & Отлично \\
	Прикладная газовая динамика & 2 & Отлично \\
	Прочность летательных аппаратов & 4 & Удовл. \\
	Теоретическая гидродинамика & 7 & Отлично \\
	Техника и методика аэродинамического эксперимента & 8 & Хорошо \\
	Управление исследованиями и разработками & 2 & Отлично \\
	Управление течением жидкости & 1 & Хорошо \\
	Физическое и численное моделирование турбулентности & 4 & Отлично \\
	Нестационарная аэродинамика летательных аппаратов & 1 & Отлично \\
	\bottomrule
	\end{tabular}}
	
\end{document}